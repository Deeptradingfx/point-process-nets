% !TeX spellcheck = fr_FR
\documentclass{beamer}
\usetheme{metropolis}
\usepackage{appendixnumberbeamer}

\metroset{
	sectionpage=none,
	progressbar=frametitle
}

\setbeamercolor{background canvas}{bg=white}

\usepackage{polyglossia}
\setdefaultlanguage{french}
\usepackage{mathtools,amsfonts,amssymb}
\usepackage{booktabs}
\usepackage{enumitem}

\usepackage{csquotes}
\usepackage[
	backend=biber,
	style=alphabetic,
	sorting=nyt
]{biblatex}
\addbibresource{../rapport/references.bib}

\usepackage{dsfont}
\usepackage{stmaryrd}
\usepackage{graphicx}
\usepackage{subcaption}

\newcommand{\RR}{\mathbb{R}}
\newcommand{\NN}{\mathbb{N}}
\newcommand{\ZZ}{\mathbb{Z}}
\newcommand{\PP}{\mathbb{P}}
\newcommand{\EE}{\mathbb{E}}
\newcommand{\Var}{\mathrm{Var}}

\DeclareMathOperator*{\argmax}{argmax} % no space, limits underneath in displays

\title{
	\textit{Processus ponctuels et réseaux de neurones récurrents}
}
\subtitle{Soutenance}
\author{
	Cheikh Fall\\
	Wilson Jallet
}

\date{11 décembre 2018}

\begin{document}
\maketitle

\begin{frame}{Introduction}
Beaucoup de phénomènes se manifestent par des événements ponctuels.
\end{frame}

\section{Modèles}

\begin{frame}{Le modèle classique: Hawkes (\citeyear{hawkes1971})}
\citeauthor{hawkes1971} introduit un modèle de processus ponctuel auto-excité
\begin{equation}
\lambda_t =
\underbrace{\mu_t}_{\substack{\text{intensité}\\\text{exogène}}} + \int_0^t \underbrace{g(t-s)}_{\text{noyau}}\cdot\, dN_s
\end{equation}
\end{frame}


\begin{frame}{Modélisation par réseau de neurones récurrent}
Le postulat de base:
\begin{equation}
\lambda_t = f(t\mid \mathcal{F}_t)
\end{equation}
avec $f(\,\cdot\,\mid \mathcal{F_{.}})$ un réseau de neurones récurrent.\pause

\noindent\textbf{Problème} \begin{itemize}
	\item Processus ponctuel en temps continu $\rightarrow$ RNN en temps continu.
	\item Choix de $f$? Comment dépendre des paramètres du réseau?
\end{itemize}

\end{frame}

\begin{frame}{Modèle LSTM \autocite{meiEisnerNeuralHawkes}}
\textbf{Idée:} \citeauthor{meiEisnerNeuralHawkes} proposent de choisir 
\begin{equation}
\lambda_t = f(W_\ell h(t))
\end{equation}
avec $h\in\RR^D, D\in\NN^*$ le \textit{hidden state}. Le temps continu est pris en compte par amortissement.
\begin{equation}\label{eq:meiEisnerHiddenStates}
\begin{aligned}
	h(t) &= o_i \odot \tanh(c(t)) \\
	c(t) &= \bar{c}_i + (c_i - \bar{c}_i)e^{-\delta_i(t - t_{i-1})}
\end{aligned} \quad t\in(t_{i-1}, t_i]
\end{equation}
\end{frame}

\begin{frame}
L'\textit{output} $o_i$, le \textit{cell state} initial $c_i$ en $t_{i-1}$ et sa valeur asymptotique $\overline{c}_i$ suivent les équations d'un réseau LSTM classique.

\textbf{Entrées du LSTM:} Pour mettre à jour les paramètres après l'événement $(t_i,k_i)$:\begin{itemize}
	\item[\textbullet] Embedding $x_i\in\RR^K$ du type d'événement
	\item[\textbullet] \textit{Hidden state} $h(t_{i})$ en $t_{i}^{-}$
	\item[\textbullet] \textit{Cell state} $c(t_{i})$ en $t_i^{-}$ (pour calculer $c_{i+1} = c(t_i^{+})$)
	\item[\textbullet] \textit{Cell state} asymptotique $\bar{c}_i$ (pour calculer $\bar{c}_{i+1}$)
\end{itemize}
\end{frame}


\begin{frame}{Variante simplifiée (RNN)}
Une version plus simple avec moins de paramètres du modèle \eqref{eq:meiEisnerHiddenStates}:
\begin{equation}
	h(t) = h_i e^{-\delta_i (t - t_{i-1})}
	\quad t\in (t_{i-1}, t_i]
\end{equation}
où $h_i = h(t_i^{+})$ est calculé en appliquant un réseau RNN simple à l'entrée $(x_i, h(t_i))$.

\end{frame}

\begin{frame}
Ces modèles ont les bonnes propriétés: $(c,h)$, puis $(\lambda_t)$, sont $(\mathcal{F}_t)$-adaptés.
\end{frame}

\begin{frame}
En théorie, les deux modèles peuvent reproduire les comportement de modèles plus classiques (e.g. Hawkes).

\begin{figure}
	\includegraphics[width=\linewidth]{../notebooks/example_rnnplot.pdf}
	\caption{Un comportement généré par un modèle RNN <<~à froid~>> (sans entraînement).}\label{fig:untrained1DRNNIntensity}
\end{figure}
	
\end{frame}

\begin{frame}
\begin{figure}
	\includegraphics[width=\linewidth]{../results/intensity_HawkesDecayRNN_1d_hidden64_20181206-234848.pdf}
	\caption{}
\end{figure}
\end{frame}

\begin{frame}[t,allowframebreaks]{References}
	\printbibliography
\end{frame}

\end{document}
