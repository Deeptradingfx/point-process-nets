% !TeX spellcheck = fr_FR
\documentclass[../main.tex]{subfiles}

\begin{document}
	
Les réseaux de neurones sont implémentés en utilisant la librairie \textsf{PyTorch} \cite{paszke2017automatic}, permettant de faire du calcul tensoriel avec différentiation automatique (pour calculer les gradients).

La log-vraisemblance d'une suite d'événements $\{(t_i,k_i)\}_i$ est donnée par la formule, démontrée dans \autocite{meiEisnerNeuralHawkes}:
\begin{equation}\label{eq:explicitLikelihood}
\mathcal{L}\left(\{(t_i,k_i)\}_i\right)
=
\sum_{i:\,t_i < T} \log\lambda^{k_i}_{t_i} - \int_0^T \underbrace{\lambda_t\cdot\mathbf{1}}_{\sum_{k=1}^K\lambda^j_t}\,dt
\end{equation}
où on a noté $\mathbf{1} = {(1,\ldots,1)}^\intercal\in\RR^K$.

Le second terme est une intégrale. Elle peut être estimée par une méthode de Monte Carlo, comme le suggèrent \citeauthor{meiEisnerNeuralHawkes}. Plus de détails sur l'estimateur utilisé sont donnés dans l'annexe \autoref{sec:algoAppendix}.

\end{document}