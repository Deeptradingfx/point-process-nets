% !TeX spellcheck = fr_FR
\documentclass[../main.tex]{subfiles}

\begin{document}

\section{Algorithmes}\label{sec:algoAppendix}

\subsection{Calcul de la log-vraisemblance}

La log-vraisemblance donnée par la relation \eqref{eq:explicitLikelihood} comporte deux termes: une somme de logarithmes d'intensités et l'intégrale $\int_0^T\lambda_t\,dt$ de l'intensité totale du processus (la somme des intensités de chaque type).

\paragraph{Estimation de l'intégrale.} Étant donné un variable aléatoire $\tau\sim\mathcal{U}([0,T])$ indépendante de la filtration $\mathds{F}$, la quantité 
\[
\bar{\Lambda} = T\lambda_\tau\cdot\mathbf 1
\]
est un estimateur non biaisé de l'intégrale. Mais il pose un problème: il faut savoir à quel intervalle $(t_{i-1}, t_i]$ appartient $\tau$, étant donné que l'expression de $\lambda_t$ est définie par morceaux. En utilisant la relation de Chasles, on peut construire l'estimateur (non biaisé) suivant:
\begin{equation}
\hat{\Lambda}^C = \sum_{i=1}^{I} \Delta t_i\lambda_{\tau_i} \cdot \mathbf{1} + (T-t_{I})\lambda_{\tau_{I+1}}\cdot\mathbf{1}
\approx
\int_0^T \lambda_t\,dt
\end{equation}
où les $\tau_i\sim\mathcal{U}([t_{i-1}, t_i])$, $\tau_{I+1}\sim\mathcal{U}([t_I, T])$ sont indépendants. Le problème de l'intervalle auquel appartient $\tau$ est levé, mais on n'a aucune garantie que $\hat{\Lambda}^C$ soit un meilleur ou pire estimateur que $\bar{\Lambda}$ en terme de variance.

\paragraph{Somme des logarithmes.} Le premier terme pose un léger problème au niveau de l'implémentation. Le calcul des intensités ${(\lambda_{t_i})}_i$ donne un tenseur \verb|intensities_at_events| de format $I\times B\times K$, avec $B$ la taille du \textit{batch} (le sous-ensemble d'entraînement que l'on est en train de traiter). Extraire les $\lambda^{k_i}_{t_i}$ ne peut pas se faire par indexation (par exemple en faisant \verb|intensity_at_events[:, :, event_types]|).

L'astuce utilisée exploite la représentation des types $k_i$ par encodage \textit{one-hot} avec des vecteurs $x_i\in{\{0,1\}}^K$. On a la relation suivante:
\[
\lambda^{k_i}_{t_i} = \sum_{k=1}^K {(\lambda_{\tau_i} \odot x_i)}_k,
\]
où $\odot$ désigne le produit terme-à-terme, qui permet d'obtenir les composantes de l'intensité que l'on veut en faisant un produit et une réduction.

\end{document}